%%%%%%%%%%%%%%%%%%%%%%%%%%%%%%%%%%%%%%%%%
% Thin Sectioned Essay
% LaTeX Template
% Version 1.0 (3/8/13)
%
% This template has been downloaded from:
% http://www.LaTeXTemplates.com
%
% Original Author:
% Nicolas Diaz (nsdiaz@uc.cl) with extensive modifications by:
% Vel (vel@latextemplates.com)
%
% License:
% CC BY-NC-SA 3.0 (http://creativecommons.org/licenses/by-nc-sa/3.0/)
%
%%%%%%%%%%%%%%%%%%%%%%%%%%%%%%%%%%%%%%%%%

%----------------------------------------------------------------------------------------
%	PACKAGES AND OTHER DOCUMENT CONFIGURATIONS
%----------------------------------------------------------------------------------------

\documentclass[article, 11pt]{article} % Font size (can be 10pt, 11pt or 12pt) and paper size (remove a4paper for US letter paper)

\usepackage[protrusion=true,expansion=true]{microtype} % Better typography
\usepackage{graphicx} % Required for including pictures
\usepackage{wrapfig} % Allows in-line images
\usepackage[margin=1.25in]{geometry}

\usepackage[scaled]{helvet} % Use the Helvetica font
\renewcommand\familydefault{\sfdefault} 
\usepackage[T1]{fontenc} % Required for accented characters

\usepackage{mathpazo} 
\usepackage[T1]{fontenc} 
\linespread{1.05} % Change line spacing here, Palatino benefits from a slight increase by default

\makeatletter
\renewcommand\@biblabel[1]{\textbf{#1.}} % Change the square brackets for each bibliography item from '[1]' to '1.'
\renewcommand{\@listI}{\itemsep=0pt} % Reduce the space between items in the itemize and enumerate environments and the bibliography

\renewcommand{\maketitle}{ % Customize the title - do not edit title and author name here, see the TITLE block below
\begin{flushright} % Right align
{\LARGE\@title} % Increase the font size of the title

\vspace{50pt} % Some vertical space between the title and author name

{\large\@author} % Author name
\\\@date % Date

\vspace{40pt} % Some vertical space between the author block and abstract
\end{flushright}
}

%----------------------------------------------------------------------------------------
%	TITLE
%----------------------------------------------------------------------------------------

\title{\textbf{New York City Economic Map}\\ % Title
Applied Urban Science \& Informatics Capstone} % Subtitle

\author{\textsc{Tong Jian, Kenneth Luna, Samuel Pollack \& Julia M. Smith} % Author
\\{\textit{M.S. Candidates, NYU Center for Urban Science + Progress}}} % Institution

\date{\today} % Date

%----------------------------------------------------------------------------------------

\begin{document}

\maketitle % Print the title section

%----------------------------------------------------------------------------------------
%	ABSTRACT AND KEYWORDS
%----------------------------------------------------------------------------------------

%\renewcommand{\abstractname}{Summary} % Uncomment to change the name of the abstract to something else

\begin{abstract}
On behalf of New York City Small Business Services and Citi's Community Development branch, the New York City Economic Map (NYCEM) provides a proof-of-concept study in support of ongoing small business development initiatives. 
NYCEM is an interactive tool that provides insight to business activity across New York City census tracts by both querying of industry statistics and clustering on key employment features.
\end{abstract}

\hspace*{3,6mm}\textit{Keywords:} New York City, economic, map, small business% Keywords

\vspace{30pt} % Some vertical space between the abstract and first section

%----------------------------------------------------------------------------------------
%	ESSAY BODY
%----------------------------------------------------------------------------------------

\section*{Objective}

The New York City Economic Map (NYCEM) has been undertaken as a capstone project following last year's development of a New York City Economic Profile. At present, the project is an internal CUSP research initiative advised by Dr. Greg Dobler and Dr. Tim Savage. In early May 2015, NYCEM advisors met with representatives from Citi's Community Development brand and New York City Small Business Services (SBS). These two stakeholders have launched a historic partnership which seeks to identify and target minority-owned businesses to best extend existing and newly developed SBS services. The purpose of the meeting was to gauge the partnership's interest in NYCEM and understand which research direction would be most helpful. The partners have expressed interest in identifying and profiling businesses in minority and low-income neighborhoods. These findings will ideally inform their outreach strategy and enhance take-up of Small Business Services resources.
\setlength{\parskip}{6pt}
Below is the low-to-moderate income (LMI) criteria that Citi Community Development
is bound to in their consideration of small business development initiatives:
\setlength{\parskip}{6pt}
\begin{enumerate}
\item An individual is considered "low-to-moderate income" (LMI) if his or her household Income is \$55,113 or below, which is less than 80 percent of the New York-Jersey City-
White Plains, NY-NJ Metropolitan Statistical Areas (MSA code 35614) Federal
Financial Institutions Examination Council (FFIEC) estimated 2014 Median Family
Income of \$68,900.
\item "A small business meets the 'economic development' criteria if it earns \$1 million or less in annual gross revenue and one of the following two criteria is also true:
The business supports permanent job creation/retention/improvement for low-to-moderate (LMI) income individuals and/or the business is located in a low-to-moderate (LMI) income census tract, or is a government-targeted redevelopment area."
\end{enumerate}
\setlength{\parskip}{6pt}
In responding to Citi's request for proof-of-concept studies, the team sought to develop a queryable database of small business activity in New York City where information is translated visually onto an on an interaction map. The database includes revenue, employment, and tenure figures across New York City census tracts. Relying heavily on public data, this project supports New York City's open data and analytics initiatives.
\setlength{\parskip}{6pt}
This document details NYCEM's trajectory from inception in April 2015 to present, July 2015. The project was submitted to advisors Dr. Greg Dobler and Dr. Tim Savage of the New York University Center for Urban Science + Progress on July 24, 105.

%------------------------------------------------

\section*{Context}

Beyond conveying to Citi the results of their queries at the census tract level, the team has sought to understand what measures have come to quantify business success. This conversation was primarily informed by related academic texts.

In 1990, economists Steven J. Davis and John Haltwanger from the University of Chicago Graduate School of Business and the University of Maryland summarizes baseline workforce trends, such as the propensity for job separation following a year of work and routine turnover within an establishment. The team noted that these trends have vast macroeconomic impact when experienced across all firms. The authors found that costs associated with churn, hiring, and displacement all posed significant friction for individual firms. \cite{Davis} This friction can largely be credited for more robust market dynamics in periods of flux-up. These findings have become increasingly relevant to today's start-up driven market, where individual business interests catalyze widespread hiring sprees but often result in buy-outs or closures in the wake of fierce industry competition.

In a further attempt to understand the impact of firm size on the economy, John Haltiwanger collaborated with Ron S. Jarmin and Javier Miranda in 2013 to investigate a commonly held belief, which largely impacts public policy, that small businesses employ the most employees. Using the Census' Longitudinal Business Database, there was a subtle inverse relationship between firm size and net growth, which supports this belief. However, once controlling for firm age, this relationship disappeared and in some cases reversed (which can be attributed to high separation at small firms.) The team found that firms over 10 years old and "have more than 500 workers account for about 45 percent of all jobs in the U.S. private sector." These mature firms "account for almost 40 percent of job creation and destruction" and resulting job creation and description across employer size cohorts "roughly equaled their share of total employment." An exception, however, is that start-ups "account for only 3 percent of employment but almost 20 percent of gross job creation." \cite{Haltiwanger} These findings seem to herald the importance of start-ups and the catalytic potential in fostering their success.

To understand this diversity of findings, a follow-up publication by John Haltiwanger, Ryan Decker, Ron Jarmin, and Javier Miranda from 2014 attempts to quantify the impact of start-ups' employment, and find the resulting analysis to be incredibly noisy. Their most recent findings illustrate that many start-ups do not succeed, yet the few that do contribute disproportionately in terms of employment and revenue. These baseline findings complicate policy recommendations and enactments which seek to prioritize the growth of potentially high-impact new companies. The primary take away is that start-ups initially hire disproportionately, and thus their failures are more pronounced given frequent firm turnover amongst this cohort. \cite{Decker} As a result of these findings, our team seeks to prioritize business tenure as the primary indicatory of long-term business success, and resultantly more stable employment.

%------------------------------------------------

\section*{Data Sources}

Data has been pulled from four primary sources, three publicly available and one commercially available. Demographic data originates from the most recent American Community Survey, which includes relevant information such as household income, race, and geography to the census tract level. Employment tenure and firm age data was procured from the Census Longitudial Employer-Household Dynamics (LEHD) Original-Destination Employment Statistics (LODES). Private data for this project has been procured by NYU from ReferenceUSA, a data provider which has collected and tabulated survey responses related to business profiles with employee count and associated revenue to a very granular level - the exact address of each surveyed business is included in the data set. All survey figures were compared to the Census' Zip Code Business Patterns in order to gauge the degree to which the survey deviated from the known universe of businesses defined by the Census.

%------------------------------------------------

\section*{Data Limitations}

The limitations of each data set are apparent: open data does not reach the level of granularity needed to produce a map at the block or lot level, and the private data, though robust in fields, is lacking in the number of businesses surveyed. Because many businesses do not have the time or inclination to respond, and those that do respond are self-reporting their revenue figures, there is a high chance that the responses are skewed in favor of high-revenue practices with administration support. In an effort to quantify variance between RefUSA - a business survey - and the Census' Zip Code Business Patterns, the team devised the following plot to demonstrate where employee counts varied significantly between the two sources. Unfortunately, Unfortunately, many zip codes reflected percentage differences beyond the central jenk, which demonstrates significant variation between sources. Low response rates and liberal estimates on RefUSA surveys are our hypothesized cause, and as seen in the map below, these poor responses were generally seen in less affluence neighborhoods such as Central Brooklyn, Eastern Queens, and Washington Heights.

%------------------------------------------------

\section*{Data Exploration}

TBD


%------------------------------------------------

\section*{Analytical Methods}

Initial efforts concentrated on understanding nuances between publicly available data sets and their role within the overarching project objectives. As seen in the NYCEM Deliverables Inventory, the team first sought to understand the universe of businesses and their unique employment count and location features via the Census Zip Code Business Patterns. All work and exploratory figures are available publicly on the team Wiki at http://github.com/gdobler/nycem.

Data wrangling efforts for construction of the database have concentrated on aligning our available data sources in order to compile a single table that will inform future querying and geographic representation. The first filter required removing businesses in RefUSA that were not located within the five boroughs. RefUSA data contained several businesses with latitudes and longitudes that place them in eastern Long Island and parts of New Jersey. After removing non-NYC based businesses, we grouped the data based on SICD, PNATITL, county/borough, and census tract and extrapolated the sum and median of both annuals sales volume (SLSVDT) and employee count (EMPSDT). Median was used instead of the average to address several businesses that do not fall within the definition of small businesses. For example, a business such as Bank of America with several hundred millions of dollars in annual revenue and thousands of employees far exceed the threshold of small businesses. Also, we decided to continue our analysis at the census tract level versus the census block level as Citigroup provided us with a minimum household income of \$55,000.

The resulting dataframe (df) provides us with a hierarchy of summary statistics at the census tract level. The PNATITL code is a broader category than the SICD code. The dataset in this form contained 4,500+ SICD codes grouped into 940+ PNATITL codes. Unfortunately, we felt that this level of granularity would be too much for our use-case, therefore we appended one of the 24 NAICS codes to each PNATITL-SICD combination. This data structure will allow us to present users with higher-level categorizations, and iteratively work their way down to the appropriate business category.

After merging the NAICS codes, we proceeded to append the shapefile polygons for the census tract linked to each NAICS-PNATIL-SICD combination. The first step was reading the NYC census tract shapefiles using GeoPandas. The dataframe loaded via GeoPandas contained the borough and census tract code for each polygon; these two attributes were used to append the polygon points. This dataframe contains over 150 thousand observations for each year between 2010 and 2014 and with the inclusion of the polygon points, is too large to convert into a geojson file. To work around this issue, we decided to create hierarchy of directories based on the NAISC, PNATITL, and SICD codes. This kept our Mapbox visualization from having to load and parse an extremely large file which can be too expensive for the front-end. For example, mobile food trucks would fall into the following hierarchical directory structure: ~/2010/ACCOMMODATION AND FOOD SERVICES/MOBILE FOOD SVCS. Another example would be vocational computer training courses: ~/2010/EDUCATIONAL SERVICES/COMPUTER TRAINING. Again, the hierarchy of data is first the year, NAICS, SICD, followed by the geojson file for SICD code.

The geojson files contains polygon points for each census tract but also other attributes specific to that business category in that area. The following is a list of all attributes and their definitions which are provided in the queryable database:

\begin{itemize} 
\item SLSVDT\_SUM: Aggregate sales volume by business type in census tract (CT)
\item SLSVDT\_MED: Median sales volume by specific business type in CT
\item EMPSDT\_SUM: Aggregate number of employees by specific business type in
CT
\item EMPSDT\_MED: Median number of employees per business for specific business
type in CT
\item count\_SUM: Number of businesses to fall in specific business type in CT
\item Shape\_Area: Area of CT in square feet
\item Geo\_COUNTY: Borough/county code
\item SHP\_BOROUGH: Borough/county name
\item SHP\_CENSUS\_TRACT: CT number
\item NTAName: Neighborhood name
\item NAICS\_LABEL: North American Industry Classification System code
\item PNACODE: Primary NAICS code
\item PNATITL: Primary NAICS Title label
\item PRMSIC: Standard Industrial Classification code
\item SICD: Standard Industrial Classification
\item lmi\_ct: Median Household Income from ACS-5 Year (07-11) ***
\item LMI: Column at \$52,259 will be added to the final database in order to facilitate
the parsing of census tracts to those below this benchmark, shown in red below.
\end{itemize}


%------------------------------------------------

\section*{Clustering}

TBD


%------------------------------------------------

\section*{Mapping Visualization}

Upon running the mapping visualization (available at http://github.com/gdobler/nycem) locally, the user is able to select queryable census tracts and view summary statistics from the database on the side panel. When the user clicks on either the side panel or the census tract within view, a pop-up in generated that provides relevant employment tenure statistics from LODES. Beyond querying, this LODES information has been clustered using k-means in order to identify census tracts that have experienced similar employment patterns.
\\\\
The map design and data layering is all created using Mapbox and the Leaflet API. For the design aspect, we took a template from Mapbox and added custom design features. The database has be coded as geojsons which are read into the map using a function from Leaflet. Using Javascript/JQuery, sub-sections of the HTML page are selected in order to store user clicks into global variables. These global variables are then concatenated to form the URL and directory where the geojson files are stored. Additional map functionality is provided by HTML and CSS. The maps drop-down menus are powered by Twitter Bootstrap, which allows for custom design.

%------------------------------------------------

\section*{Future Work}

[TBD]

%------------------------------------------------

%\section*{Works Cited}

\begin{thebibliography}{100}
\bibitem{Davis} Davis, S., Haltiwanger, J., `Gross Job Creation, Gross Job Destruction, and Employment Reallocation," \emph{University of Chicago and University of Maryland}, 1990.
\bibitem{Haltiwanger} Haltiwager, J., Jarmin, R., and Miranda, J., ``Who Creates Jobs? Small vs. Large vs. Young," \emph{The Review of Economics and Statistics}, 2013.
\bibitem{Decker} Decker, R., Haltiwanger, J., Jarmin, R., and Miranda, J. ``The Role of Entrepreneurship in U.S. Job Creation and Economic Dynamism," \emph{Journal of Economic Perspectives}, pp. 3-24, 2014.
\end{thebibliography}

%------------------------------------------------

\section*{Appendix}

[ADD UPDATED VISUALS]
\\\
[ADD DATA SOURCE METHODOLOGY] 

%----------------------------------------------------------------------------------------

\end{document}